\documentclass{article}
\usepackage[utf8]{inputenc}
\usepackage{amsmath}
\usepackage{amssymb}
\usepackage{amsfonts}
\usepackage{geometry}
\usepackage{tabularx}
\usepackage{booktabs}
\usepackage{bm}

\geometry{margin=1in}

\title{Dynamic Programming Formulation for Energy-Economy Integrated Assessment Model}
\author{}
\date{}

\begin{document}

\maketitle

\section{Model Overview}

The dynamic programming formulation solves a stochastic optimal control problem for an integrated energy-economy system using the Bellman equation approach with backward induction. The model employs a three-dimensional state space with vintage capital accumulation and endogenous energy service demand determination.

\section{State Space Definition}

The model operates over discrete time periods $t \in \{1, 2, \ldots, T\}$ corresponding to years $\{2020, 2030, 2040, 2050, 2060, 2070, 2080\}$. The state space is three-dimensional:

\begin{align}
\mathcal{S}_t = \{(K_t, Y_t, PRODENE_t) : K_t \in [K_{\min}, K_{\max}], Y_t \in [Y_{\min}, Y_{\max}], PRODENE_t \in [P_{\min}, P_{\max}]\}
\end{align}

where:
\begin{itemize}
\item $K_t$ is total capital stock (trillion USD)
\item $Y_t$ is total production capacity (trillion USD) 
\item $PRODENE_t$ is energy service capacity (production energy units)
\end{itemize}

Grid bounds and discretization:
\begin{align}
K_{\min} &= 50.0, \quad K_{\max} = 600.0, \quad n_K = 30 \\
Y_{\min} &= 30.0, \quad Y_{\max} = 300.0, \quad n_Y = 20 \\
P_{\min} &= 75.0, \quad P_{\max} = 300.0, \quad n_{PRODENE} = 15
\end{align}

\section{Production Technology}

The economy employs a nested CES production function with vintage capital structure:

\begin{align}
Y_t^N &= \left[a \cdot (K_t^N)^{\rho \kappa} \cdot (L_t^N)^{\rho(1-\kappa)} + b \cdot (ELEC_t^N)^{\rho \varepsilon} \cdot (NELE_t^N)^{\rho(1-\varepsilon)}\right]^{1/\rho}
\end{align}

where:
\begin{itemize}
\item $Y_t^N$ is new vintage production capacity
\item $K_t^N$ is new vintage capital 
\item $L_t^N$ is new vintage labor supply
\item $ELEC_t^N, NELE_t^N$ are new electricity and non-electricity energy services
\item $\rho = -0.233$ is the CES transformation parameter (elasticity of substitution $\sigma = 0.3$)
\item $\kappa = 0.28$ is the capital value share parameter
\item $\varepsilon = 0.42$ is the electricity value share parameter
\item $a, b$ are calibrated efficiency parameters
\end{itemize}

\section{Vintage Capital Dynamics}

Both capital and production capacity follow vintage accumulation with depreciation:

\begin{align}
K_{t+1} &= K_t (1-\delta)^{\Delta t} + K_t^N \\
Y_{t+1} &= Y_t (1-\delta)^{\Delta t} + Y_t^N \\
PRODENE_{t+1} &= PRODENE_t (1-\delta)^{\Delta t} + NEWENE_t
\end{align}

where:
\begin{itemize}
\item $\delta = 0.05$ is the annual depreciation rate
\item $\Delta t = 10$ is the period length in years
\item $K_t^N = \Delta t \cdot I_t$ relates new capital to investment
\item $NEWENE_t = ELEC_t^N + NELE_t^N$ is total new energy services
\end{itemize}

\section{Energy Service Determination}

Given a target new production $Y_t^N$, the optimal energy service allocation is determined analytically. From the first-order conditions of the CES production function, the optimal energy ratio is:

\begin{align}
r^* = \frac{ELEC_t^N}{NELE_t^N} = \left(\frac{\varepsilon}{1-\varepsilon} \cdot \frac{p_{NELE}}{p_{ELEC}}\right)^{1/(1-\rho)}
\end{align}

where $p_{ELEC} = 0.0567$ and $p_{NELE} = 0.020$ are calibrated energy prices.

Given the target production $Y_t^N$ and capital-labor input $K_t^N, L_t^N$, the required energy term is:

\begin{align}
E_{term} = (Y_t^N)^\rho - a \cdot (K_t^N)^{\rho \kappa} \cdot (L_t^N)^{\rho(1-\kappa)}
\end{align}

The analytical solution for non-electricity energy is:

\begin{align}
NELE_t^N = \left(\frac{E_{term}}{b \cdot (r^*)^{\rho \varepsilon}}\right)^{1/\rho}
\end{align}

And electricity energy:
\begin{align}
ELEC_t^N = r^* \cdot NELE_t^N
\end{align}

\section{Energy Cost Surrogate Model}

Energy system costs are approximated using a time-aware autoregressive surrogate model:

\begin{align}
EC_t = \beta_0 + \beta_1 EC_{t-1} + \beta_2 PHYSENE_{ELEC,t} + \beta_3 PHYSENE_{NELE,t} + \beta_4 t + \beta_5 (PHYSENE_{total,t} \cdot t)
\end{align}

where physical energy demands include autonomous energy efficiency improvement (AEEI):

\begin{align}
PHYSENE_{ELEC,t} &= PRODENE_{ELEC,t} \cdot AEEI_{ELEC,t} \\
PHYSENE_{NELE,t} &= PRODENE_{NELE,t} \cdot AEEI_{NELE,t} \\
AEEI_{s,t} &= \prod_{j=1}^{t} (1 - \alpha_{s,j})^{\Delta t}
\end{align}

with $\alpha_{s,j} = 0.02$ for both electricity and non-electricity sectors.

The surrogate model coefficients are:
\begin{align}
\boldsymbol{\beta} = [268.278, -0.0283, 106.981, 17.496, -58.096, 0.157]
\end{align}

Energy demands are projected based on GDP growth:
\begin{align}
PHYSENE_{s,t} = PHYSENE_{s,base} \cdot \left(\frac{GDP_t}{GDP_{base}}\right)^{0.7}
\end{align}

\section{Bellman Equation}

The value function satisfies the Bellman equation:

\begin{align}
V_t(K_t, Y_t, PRODENE_t) = \max_{I_t, Y_t^N} \left\{ UDF_t \cdot u(C_t) \cdot \Delta t + \beta \cdot V_{t+1}(K_{t+1}, Y_{t+1}, PRODENE_{t+1}) \right\}
\end{align}

subject to the budget constraint:
\begin{align}
C_t = Y_t - I_t - EC_t
\end{align}

and state evolution equations given above.

The utility function employs logarithmic preferences:
\begin{align}
u(C_t) = \ln(C_t)
\end{align}

Utility discount factors account for period-specific discounting:
\begin{align}
UDF_t = \left(\frac{1}{1 + \delta_{rate}}\right)^{t-1}
\end{align}

where $\delta_{rate} = 0.05$ is the social discount rate.

\section{Terminal Condition}

For the terminal period $T$ (year 2080), the model applies a finite-time correction to account for post-terminal utility:

\begin{align}
V_T(K_T, Y_T, PRODENE_T) = UDF_T \cdot \ln(C_T) \cdot \left(\Delta t + \frac{1}{\gamma_{T}}\right)
\end{align}

where $\gamma_T = g_T + \delta_{rate}$ combines the terminal growth rate and discount rate.

Terminal investment follows the steady-state condition:
\begin{align}
I_T = K_T \cdot (g_T + \delta)
\end{align}

\section{Labor Supply Evolution}

Labor supply grows exogenously with separate accounting for vintage labor:

\begin{align}
L_t &= \prod_{j=1}^{t} (1 + g_{j})^{\Delta t} \\
L_t^N &= \begin{cases}
L_t & \text{if } t = 1 \\
L_t - L_{t-1} \cdot (1-\delta)^{\Delta t} & \text{if } t > 1
\end{cases}
\end{align}

where $g_t$ are period-specific growth rates.

\section{Analytical First-Order Conditions}

From the Bellman problem
\begin{align}
V_t(K_t, Y_t, P_t) = \max_{I_t, Y_t^N} \left\{ UDF_t \cdot \ln(C_t) \cdot \Delta t + \beta \cdot V_{t+1}(K_{t+1}, Y_{t+1}, P_{t+1}) \right\}
\end{align}

with
\begin{align}
C_t &= Y_t - I_t - EC_t, \\
K_{t+1} &= K_t(1-\delta)^{\Delta t} + \Delta t \cdot I_t, \\
Y_{t+1} &= Y_t(1-\delta)^{\Delta t} + Y_t^N, \\
P_{t+1} &= P_t(1-\delta)^{\Delta t} + E_t^N,
\end{align}

the two first-order conditions are:

\subsection{FOC with respect to $I_t$}

\begin{align}
UDF_t \cdot \frac{1}{C_t} \cdot (-1) \cdot \Delta t + \beta \cdot \frac{\partial V_{t+1}}{\partial K_{t+1}} \cdot \Delta t = 0
\end{align}

This is the standard Euler equation for optimal investment.

\subsection{FOC with respect to $Y_t^N$}

\begin{align}
UDF_t \cdot \frac{1}{C_t} \cdot \left(-\frac{\partial EC_t}{\partial Y_t^N}\right) \cdot \Delta t + \beta \left[\frac{\partial V_{t+1}}{\partial Y_{t+1}} + \frac{\partial V_{t+1}}{\partial P_{t+1}} \cdot \frac{\partial P_{t+1}}{\partial Y_t^N}\right] = 0
\end{align}

where $\frac{\partial EC_t}{\partial Y_t^N}$ comes from the AR-surrogate model and $\frac{\partial P_{t+1}}{\partial Y_t^N}$ follows from the analytical energy allocation (the $r^*$-formula).

These conditions are fully "plug-and-chug" once the partial derivatives $\frac{\partial V_{t+1}}{\partial K_{t+1}}$, $\frac{\partial V_{t+1}}{\partial Y_{t+1}}$, and $\frac{\partial V_{t+1}}{\partial P_{t+1}}$ are evaluated through interpolation/approximation.

\section{Envelope Theorem Conditions}

For an interior optimum in a smooth, concave dynamic programming problem, the envelope theorem allows us to differentiate the Bellman equation with respect to state variables while ignoring the indirect effect through optimally chosen controls.

Let
\begin{align}
V_t(K_t,Y_t,P_t) = \max_{I_t,Y^N_t}\Bigl\{\,\underbrace{UDF_t\,\ln(C_t)\,\Delta t}_{f(C_t)} + \beta\,V_{t+1}(K_{t+1},Y_{t+1},P_{t+1})\Bigr\}
\end{align}

where the state transitions are as defined above.

\subsection{Envelope with respect to $K_t$}

\begin{align}
\frac{\partial V_t}{\partial K_t} = \underbrace{\frac{\partial f}{\partial K_t}}_{0} + \beta \left[V_{K_{t+1}} \cdot \frac{\partial K_{t+1}}{\partial K_t}\right] = \beta \cdot (1-\delta)^{\Delta t} \cdot V_{t+1,K}
\end{align}

Here $\frac{\partial f}{\partial K_t} = 0$ because current payoff $f$ depends on $K_t$ only through $I_t$, and the envelope theorem tells us to hold $I_t$ fixed.

\subsection{Envelope with respect to $Y_t$}

\begin{align}
\frac{\partial V_t}{\partial Y_t} = \underbrace{UDF_t \cdot \frac{1}{C_t} \cdot \Delta t}_{f_{Y_t}} + \beta \left[V_{t+1,Y} \cdot \frac{\partial Y_{t+1}}{\partial Y_t}\right] = UDF_t \cdot \frac{\Delta t}{C_t} + \beta \cdot (1-\delta)^{\Delta t} \cdot V_{t+1,Y}
\end{align}

Since $\frac{\partial C_t}{\partial Y_t} = 1$ and $\frac{\partial Y_{t+1}}{\partial Y_t} = (1-\delta)^{\Delta t}$.

\subsection{Envelope with respect to $P_t$}

\begin{align}
\frac{\partial V_t}{\partial P_t} = \underbrace{\frac{\partial f}{\partial P_t}}_{0} + \beta \left[V_{t+1,P} \cdot \frac{\partial P_{t+1}}{\partial P_t}\right] = \beta \cdot (1-\delta)^{\Delta t} \cdot V_{t+1,P}
\end{align}

\subsection{Regularity Conditions}

The envelope theorem requires the following assumptions:
\begin{enumerate}
\item \textbf{Interior solution}: neither $I_t$ nor $Y^N_t$ lie on a constraint boundary.
\item \textbf{Differentiability}: $u(C)$, the surrogate $EC_t$, and the transition maps are continuously differentiable.
\item \textbf{Concavity}: the Bellman operator objective is concave in $(I_t,Y^N_t)$, ensuring a unique interior maximizer.
\item \textbf{Nonbinding state constraints}: any bounds on $K_t,Y_t,P_t$ are not active at the optimum.
\end{enumerate}

Under these conditions, the envelope derivatives can be safely used when writing FOCs and implementing either a Newton-type solver or a projection method for $V$.

\section{Numerical Solution Algorithm}

The model is solved using backward induction with discrete state space approximation:

\textbf{Step 1:} Initialize terminal value function $V_T(K_T, Y_T, PRODENE_T)$ using terminal conditions.

\textbf{Step 2:} For $t = T-1, T-2, \ldots, 1$:
\begin{enumerate}
\item For each state $(K_t, Y_t, PRODENE_t)$ on the discrete grid
\item Search over feasible investment levels $I_t \in [I_{\min}, I_{\max}]$
\item Search over feasible production targets $Y_t^N \in [Y^N_{\min}, Y^N_{\max}]$
\item Calculate energy allocation using analytical solution
\item Evaluate energy costs using surrogate model
\item Compute continuation value using interpolation
\item Select optimal policy maximizing current + continuation value
\end{enumerate}

\textbf{Step 3:} Simulate forward using optimal policy functions to generate economic trajectories.

\section{Calibrated Parameters}

The CES production function is calibrated to base year (2020) data:

\begin{center}
\begin{tabular}{lr}
\toprule
Parameter & Value \\
\midrule
Base year GDP ($Y_0$) & 71.0 trillion USD \\
Base year capital ($K_0$) & 198.8 trillion USD \\
Base year consumption ($C_0$) & 51.0 trillion USD \\
Base year investment ($I_0$) & 14.94 trillion USD \\
Electricity demand ($ELEC_0$) & 22.6 PWh \\
Non-electricity demand ($NELE_0$) & 87.3 PWh \\
Capital coefficient ($a$) & Calibrated \\
Energy coefficient ($b$) & Calibrated \\
\bottomrule
\end{tabular}
\end{center}

Economic parameters:
\begin{center}
\begin{tabular}{lr}
\toprule
Parameter & Value \\
\midrule
Annual discount factor ($\beta$) & $0.95^{10}$ \\
Annual depreciation rate ($\delta$) & 0.05 \\
Social discount rate ($\delta_{rate}$) & 0.05 \\
Period length ($\Delta t$) & 10 years \\
Elasticity of substitution ($\sigma$) & 0.3 \\
Capital value share ($\kappa$) & 0.28 \\
Electricity value share ($\varepsilon$) & 0.42 \\
\bottomrule
\end{tabular}
\end{center}

\section{Model Output}

The dynamic programming solution yields optimal time paths for:
\begin{itemize}
\item Capital stock $\{K_t\}_{t=1}^T$
\item Production capacity $\{Y_t\}_{t=1}^T$ 
\item Consumption $\{C_t\}_{t=1}^T$
\item Investment $\{I_t\}_{t=1}^T$
\item Energy service demands $\{ELEC_t, NELE_t\}_{t=1}^T$
\item Energy costs $\{EC_t\}_{t=1}^T$
\item Total discounted utility $U = \sum_{t=1}^T UDF_t \cdot u(C_t) \cdot \Delta t$
\end{itemize}

The formulation provides a computationally tractable approach to solving the integrated energy-economy optimization problem while maintaining consistency with the underlying MESSAGE-MACRO framework.

\end{document}